\begin{frame}{Conclusions}
	In this work:
	\begin{enumerate}
		\item We have seen how games naturally decompose in \textcolor{colorarena}{\emph{arenas}} and \textcolor{coloragents}{\emph{selection functions}}, with \textcolor{coloragents}{\emph{reparametrisations}} playing a key role
		\item This allows to handle \textbf{long-range correlations} in players' behaviour
		\item Hence we can easily map extensive form trees to open games with agency
	\end{enumerate}
\end{frame}

\begin{frame}{Future directions}

	There remain some open questions related to the EF$\to$OG translation:
	\begin{enumerate}
		\item Can we simplify the resulting game using topological moves?\\
		\textcolor{colornote}{e.g. translating IEF often yields OG which are $\otimes$-decomposable}
		\item Can we treat subgame-perfect equilibrium?\\
		\textcolor{colornote}{Can Escard\'o-Oliva product of selections\footnote[frame]{\cite{escardo2015bar}} be a lax monoidal structure on $\S$?}
	\end{enumerate}

	\vfill
	\onslide<2->{
		Moreover, the framework of open games (with agency) is still incomplete.
		\begin{enumerate}
			\item Selection functions do not interact much with arenas\\
			\textcolor{colornote}{Is there a better way to equip arenas with equilibrium predicates?}
			\item What kind of assignment is `$\colag{\argmax} : \colar{\Arenas} \to \Games$'?\\
			\textcolor{colornote}{An oplax Para coalgebra?} %Can we define an `Hicks laxator'?
			% \item Are equilibria representable in some sense?\\
			% \textcolor{colornote}{In other words, what's the best }
			% \textcolor{colornote}{Possible domain cat described in \cite{streufert2021category}}.
		\end{enumerate}
	}
\end{frame}
