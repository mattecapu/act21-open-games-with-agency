\begin{frame}{Conclusions}
	In this work:
	\begin{enumerate}
		\item We have seen how games naturally decompose in \emph{arenas} and \emph{selection functions}, with \emph{reparametrisations} playing a key role
		\item This allows to handle long-range correlations in players' behaviour
		\item Hence we can easily map extensive form trees to open games with agency
	\end{enumerate}
\end{frame}

\begin{frame}{Future directions}
	\begin{enumerate}
		\item Is the translation `functorial'? Possible domain cat described in \cite{streufert2021category}.
		\item Once the diagram is drawn, can we simplify it using topological moves?
		e.g. mapping `simultaneity' to $\otimes$.
		\item Infinite trees?
		\item Can we treat SPE?
		\item What kind of assignment is $\colag{\argmax} : \colar{\Arenas} \to \Games$?
		\item Can we make selections compositional?
		\item Dependent types
	\end{enumerate}
\end{frame}
