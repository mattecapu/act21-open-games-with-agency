\begin{frame}{Introduction}
	\vfill
	\begin{quotation}
		Game theory is the mathematical study of interaction among independent, self-interested agents.\\
		{\color{colornote}-- Essentials of Game Theory \cite{leyton2008essentials}}
	\end{quotation}

	Open games as a framework for compositional game theory
	(string diagram)

	Classical game theory: normal form and extensive form
	(PD in both forms)
\end{frame}

\begin{frame}{Introduction}
	In this work, we give a canonical translation from extensive form games to open games


	To do so, we introduce:

	\begin{enumerate}
		\item Open games with \emph{agency}, an improved compositional framework for games,
		\item An operator calculus for games, in particular new \emph{choice operators},
		\item \emph{Inductive data types for extensive form games} of (im)perfect information.
	\end{enumerate}
\end{frame}

% 1. Open Games with Agency
% - lenses and bidirectional information flow (= attuale)
% - straight to: arena as parametrised lens (slide 12 ma con strat (slide 23 but rename costrategies and redraw))
% - composition
% - choice
% - reparametrisation & regrouping
% - closing an arena: state and payoff
% - keep slide 18
% - keep slide 22-25 but (a) condense intro to 1 slide, (b) slide 25 on directly exemplify lens_S (or not?)
% - slide 27 add '\in \varepsilon \boxtimes \eta(k)' to the diagram
% - put slide 28 in slide 29 with def

% 2. Extensive form and translation
% - lots of pictures of trees
% - PETree & defs
% - translation: a single picture
% - an example (probably to work out 'live'?)
% - IETree & defs
% - translation: a single picture
% - an example (probably to work out 'live'?)

% 3. Future work
% - Functoriality?
% - Copy from paper
